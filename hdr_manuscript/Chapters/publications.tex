\chapter{Liste des publications}
Je publie dans le domaine de l’Intégration de données et de connaissances et dans le domaine de la bioinformatique. La plupart des publications sont indexées par :
 #### Ajouter les publies en cours d'evaluation####
\begin{itemize}
\item DBLP Computer Science Bibliography (17 entrées le 20/02/2019) : \\
http://dblp.uni-trier.de/pers/hd/l/Larmande:Pierre
\item PUBMED (13 entrées le  20/02/2019 : \\
https://www.ncbi.nlm.nih.gov/pubmed/?term=larmande+Pierre 
\end{itemize}

\vspace{0.5cm}
Mes publications sont listées ci-dessous par année de parution. Les impacts factor recensés dans cette liste sont issus des sites des journaux et mis à jour le 20/11/2018. Le rang des conférences est issu du site CORE (http://103.1.187.206/core) et mis à jour le 20/11/2018.
De nombreux articles ont été rédigés avec les doctorants et étudiants que je co-encadre. La règle pour l’ordre des noms est la suivante : le doctorant/étudiant en premier et les encadrants ou collaborateurs par ordre de taux de participation. Le dernier auteur est en général le responsable du projet. Dans le cas d'encadrement d'etudiants, il correspond au superviseur du travail.
Les conférences internationales et nationales en biologie et bioinformatique ne produisent pas toujours des proceedings. Par exemple la conférence Plant et Animal Genomes PAG rassemble plus de 3000 scientifiques depuis 20 ans sans produire de proceedings. C’est le cas également de JOBIM en France. 

\section*{Thèse}

\section*{Publications nationales avec comité de lecture}
\subsection*{Édition d’ouvrages} 
\begin{itemize}
\item [E1] Do H, Than K, Larmande P. Evaluating Named-Entity Recognition approaches in plant molecular biology. MIWAI Vietnam (Hanoi). Springer LNAI  proceedings 11248. pp 219-225 2018
\item [E2]	Ngompé GT, Venkatesan A, Hassouni N, Ruiz M, Larmande P. AgroLD API Une architecture orientée services pour l’extraction de connaissances dans la base de données liées AgroLD. Lavoisier. 2016. 21:133–58. Impact Factor: 1.046
\end{itemize}

\subsection*{Publications internationales avec comité de lecture} 

\begin{enumerate}
\item Yaw Nti-Addae, Dave Matthews, Victor Jun Ulat, Raza Syed, Guilhem Sempere, Adrien Petel, Jon Renner, Pierre Larmande, Valentin Guignon, Elizabeth Jones, Kelly Robbins. Benchmarking Database Systems for Genomic Selection Implementation. 2019. Gigascience (submitted). Impact Factor : 7.31
\item The BrAPI Consortium. BrAPI - an Application Programming Interface for Plant Breeding Applications. 2019. BioInformatics.(submitted). Impact Factor: 5.41
\item  Venkatesan A., Tagny G., El Hassouni N., Chentli I., Guignon V., Jonquet C., Ruiz M., and Larmande P. Agronomic Linked Data (AgroLD): a Knowledge-based System to Enable Integrative Biology in Agronomy. PLoS ONE 13(11): e0198270.  Impact Factor: 2.766
\item Juanillas V.M.J., Dereeper A., Beaume N., Droc G., Dizon J., Mendoza J.R., Perdon J.P., Mansueto L., Triplett L., Lang J., Zhou G., Ratharanjan K., Plale B., Haga J., Leach J.E., Ruiz M., Thomson M., Alexandrov N., Larmande P., et al. Rice Galaxy: an open resource for plant science. Giga Science. 2018 (In Press) Impact Factor: 7.31
\item	Cubry P., Tranchant-Dubreuil C., Thuillet A.C., Monat C., Ndjiondjop M.N., Labadie K., Cruaud C., Engelen S., Scarcelli N., Rhoné B., Burgarella C., Dupuy C., Larmande P., Wincker P., François O., Sabot F., and Vigouroux Y. The Rise and Fall of African Rice Cultivation Revealed by Analysis of 246 New Genomes. Curr Biol. Elsevier; 2018;28: 2274–2282.e6. Impact Factor: 9.201
\item Harper L., Campbell J., Cannon E.K., Jung S., Main D., Poelchau M., Walls R., Andorf C., Arnaud E., Berardini T., Birkett C., Cannon S., Carson J., Condon B., Cooper L., Dunn N., Farmer A., Ficklin S., Grant D., et al. AgBioData Consortium Recommendations for Sustainable Genomics and Genetics Databases for Agriculture. Database. 2018; 1–7. Impact Factor: 3.978
\item Armin Scheben A., Chan K., Mansueto L., Mauleon R., Larmande P., Alexandrov N., Wing R., McNally K., Quesneville
H., Edwards D. Progress in single access information systems for wheat and rice crop improvement. Briefing in Bioinformatics. 2018; 4:1-7 Impact Factor: 5.134
\item	Jonquet C, Toulet A, Arnaud E, Aubin E, Dzalé-Yeumo E, Emonet V, Graybeal J, Laporte M-A, Musen M, Pesce V, Larmande P. AgroPortal: an ontology repository for agronomy. Comput. Electron. Agric. 2018; 144:126–143 Impact Factor: 2.201
\item	Dzale Yeumo E, Alaux M, Arnaud E, Aubin S, Baumann U, Buche P, et al. Developing data interoperability using standards: A wheat community use case. F1000Research. 2017;6:1843.
\item	Cohen-Boulakia S, Belhajjame K, Collin O, Chopard J, Froidevaux C, Gaignard A, et al. Scientific workflows for computational reproducibility in the life sciences: Status, challenges and opportunities. Futur. Gener. Comput. Syst. 2017.75 : 284-298. Impact Factor: 2.786
\item	The South Green Collaborators. The South Green portal: a comprehensive resource for tropical and Mediterranean crop genomics. Curr. Plant Biol. 2016. 7-8 : 6-9. Impact Factor : 1.68
\item	Sempéré G, Philippe F, Dereeper A, Ruiz M, Sarah G, Larmande P. Gigwa—Genotype investigator for genome- wide analyses. Gigascience. 2016. 5:25. Impact Factor : 7.31
\item	Al-Tam, F., Adam, H., Dos Anjos, A., Lorieux, M., Larmande, P., Ghesquière, A., Jouannic, S., and H-R Shahbazkia, P-TRAP: a Panicle Traits Phenotyping Tool. 2013, BMC Plant Biology, 13:122-136. Impact Factor: 3.631
\item	Wollbrett J, Larmande P, de Lamotte F, Ruiz M. Clever generation of rich SPARQL queries from annotated relational schema: application to Semantic Web Service creation for biological databases. BMC Bioinformatics. 2013. 14:126-141.Impact Factor: 2.435
\item Lorieux M, Blein M, Lozano J, Bouniol M, Droc G, Diévart A, et al. In-depth molecular and phenotypic characterization in a rice insertion line library facilitates gene identification through reverse and forward genetics approaches. Plant Biotechnol. J. 2012;10:555–568. Impact Factor: 7.443
\item	Droc G, Périn C, Fromentin S, Larmande P. OryGenesDB 2008 update: database interoperability for functional genomics of rice. Nucleic Acids Res. 2009;37:D992-D995. Impact factor: 9.202
\item	Larmande P, Gay C, Lorieux M, Périn C, Bouniol M, Droc G, Sallaud C, Perez P, Barnola I, Biderre-Petit C, Martin J, Morel JB, Johnson AA, Bourgis F, Ghesquière, A, Ruiz M, Courtois B, Guiderdoni E. Oryza Tag Line, a phenotypic mutant database for the Genoplante rice insertion line library. Nucleic Acids Res. 2008 Jan; 36(Database issue):D1022-D1027. Impact factor: 9.202
\item	Droc G, Ruiz M, Larmande P, Pereira A, Piffanelli P, Morel JB, et al. OryGenesDB: a database for rice reverse genetics. Nucleic Acids Res. 2006;34:D736–D740. Impact factor: 9.202
\item Sallaud C., Gay C., Larmande P., Bès M., Piffanelli P., Piégu B., Droc G., Regad F., Bourgeois E., Meynard D., Périn C., Sabau X., Ghesquière A., Delseny M., Glaszmann J.C., Guiderdoni, E. (2004) High throughput T-DNA insertion mutagenesis in rice : A first step towards in silico reverse genetics. Plant J. 2004 Aug; 39(3):450-64  Impact Factor: 5.468 
\item	Pugh T., Fouet O., Risterucci A.M., Brottier P., Abouladze M., Deletrez C., Courtois B., Clement D., Larmande P., N'Goran J.A., Lanaud C., A new cacao linkage map based on codominant markers: development and integration of 201 new microsatellite markers. Theor Appl Genet. 2004. 108(6):1151-61. 2004. Impact Factor; 3.900
\item	Sallaud C., Meynard D., van Boxtel J., Gay C., Bes M., Brizard J.P., Larmande P., Ortega D., Raynal M., Portefaix M., Ouwerkerk P.B., Rueb S., Delseny M., Guiderdoni E.,  Highly efficient production and characterization of T-DNA plants for rice (Oryza sativa L.) functional genomics. Theor Appl Genet, 2003; 106 :1396-1408. Impact Factor; 3.900
\end{enumerate}

\subsection*{Communications internationales avec comité de lecture}
\begin{itemize} 
\item [C1] Larmande P. The AgroLD project A Knowledge Graph-based Semantic Database for rice functional genomics. Oral presentation at International Symposium on Rice Functional Genomics ISRFG 2018. Tokyo (Japan) 2 pages.
\item [C1b] Do H., Than K., and Larmande P. Evaluating Named-Entity Recognition approaches in plant molecular biology. MIWAI 2018.Proceedings LNCS AI; 2018. 14 Pages
\item [C1c] Do H., Than K., and Larmande P. Comparative NER approaches in plant molecular biology. CiCling 2018. Proceedings RCS ; 2018 (In Press) 7 Pages
\item [C1d]	Larmande P., El Hassouni N. , Venkatesan A., Tagny G., Ruiz M. The Agronomic Linked Data project (AgroLD) a knowledge network platform for rice. Oral presentation at International Symposium on Rice Functional Genomics ISRFG 2017. Sewon (Korea). 2 pages
\item [C2]	Venkatesan A., Tagny G., El Hassouni N., Ruiz M., Larmande P. The Agronomic Linked Data project. Computer demo at Plant and Animal Genomes Conference PAG 2017. San Diego, (USA). 2 pages
\item [C3]	 Sempere G., Phillippe  F., Dereeper A., Ruiz M, Sarah G. and Larmande P. Gigwa: Genotype Investigator for Genome Wide Analyses. Computer demo at Plant and Animal Genomes Conference PAG 2017. San Diego, (USA). 2 pages
\item [C4]	Zevio S., El Hassouni N., Ruiz M. and Larmande P. AgroLD indexing tools with ontological annotations. Poster at Semantic Web for Life Science SWAT4LS 2016. Cambridge (UK) 2 pages
\item [C5]	Jonquet C, Toulet A, Arnaud E, Aubin S, Yeumo ED, Emonet V, Graybeal J, Musen MA, Pommier C, Larmande P.   Reusing the NCBO BioPortal technology for agronomy to build AgroPortal. Proceedings International Conference on Biomedical Ontology and BioCreative ICBO BioCreative 2016. CEUR Vol. 1747 Corvalis (USA) 6 pages.
\item [C6]	Le Ngoc L, Tireau A, Venkatesan A, Neveu P, Larmande P. Development of a knowledge system for Big Data: Case study to plant phenotyping data. Proceedings. 6th Int. Conf. Web Intell. Min. Semant. WIMS 2016, Nimes, Fr. June 13-15, 2016. ACM. p. 27:1-:9.. Nimes (France) 
\item [C7]	Larmande P. Ontology-based services and knowledge management in the Agronomic Domain. Oral presentation at the 6th Research Data Alliance Conference RDA’2015. Paris (France). 2 pages.
\item [C8]	Pierre Larmande. Gigwa - Genotype Investigator for Genome Wide Analyses. Computer demo at Plant and Animal Genomes Conference PAG 2015. San Diego, (USA). 2 pages.
\item [C9]	Larmande P, Venkatesan A, Jonquet C., Ruiz M. Sempere G., Valduriez P. Enabling knowledge management in the Agronomic Domain . Computer demo at Plant and Animal Genomes Conference PAG 2015. San Diego, (USA). 2 pages.
\item [C10]	Larmande P., Mougenot I., Jonquet C., Libourel T., Ruiz M., Arnaud E. Proceedings Semantics for Biodiversity Workshop. ESWC 2013. Montpellier (France) 4 pages.
\item [C11]	Maillol V, Bacilieri R, Sidibe Bocs S, Boursiquot J, Carrier G, Dereeper A, Droc G, Fleury C, Larmande P, Lecunff L, Péros JP, Pitollat B, Ruiz M, Sarah G, Sempéré G, Summo M, This P, and Dufayard JF. Role of Galaxy in a bioinformatic plant breeding platform. Poster at the Galaxy Community Conference 2012. Chicago (USA) 4 pages.
\item [C12]	Julien Wollbrett, Pierre Larmande and Manuel Ruiz. Towards Automatic Generation of Semantic Web services for relational Databases. Oral presenation at the International Workshop on Resources Discovery in conjunction with ESWC 2011. Heraclion (Greece) 6 pages.
\item [C13]	Larmande, P. Orylink: A Personalized Integrated System for Functional Genomic Analysis. Computer demo at Plant and Animal Genomes Conference PAG 2009. San Diego, (USA). 2 pages.
\item [C14]	Fromentin S., Droc G. and Larmande P. A personalized integrated system for rice functional genomic analysis. Poster at the 5th International Symposium of Rice Functional Genomics ISRFG 2007. Tsukuba (Japon). 2 pages.
\item [C15]	Fromentin S., Droc G. and Larmande P. A personalized, integrated system for rice functional genomics. Poster at Network Tools and Applications in Biology NETTAB 2007, Pise (Italy) 4 pages.
\end{itemize} 

\subsection*{Communications nationales avec comité de lecture }
\begin{itemize} 
\item [C16]	 Larmande P. Gigwa: Genotype Investigator for Genome Wide Analyses. JOBIM 2018. Marseille. 2 pages
\item [C16a]	Larmande P. Exposing French agronomic resources as Linked Open Data. Oral Presentation Conference Francophone d’ingénierie des connaissances, IC 2016. Montpellier (France) 6 pages.
\item [C17]	Chentli I, Larmande P., Todorov K. Construction d’un gold standard pour les données agronomiques. Poster Conference Francophone d’Ingénierie des Connaissances, IC 2016. 251-254. Montpellier (France) 4 pages.
\item [C18]	Venkatesan A, El Hassouni N, Philippe F, Pommier C, Quesneville H, Ruiz M and Larmande P. Towards efficient data integration and knowledge management in the Agronomic domain. Presentation orale ā la Conference Francophone d’ingenierie des connaissances, Rennes, 2015. 6 pages.
\item [C19]	Robakowska Hyzorek D., Mirouze M., Larmande P. Integration and Visualization of Epigenome and Mobilome Data in Crops. Poster aux Journées ouvertes pour la Biologie, l’informatique et les Mathematiques JOBIM 2016. Lyon (France). 2 pages.
\item [C20]	Le Ngoc L., Jouannic S. and Larmande P. Développement d'un outil générique d'indexation pour optimiser l'exploitation de données biologiques. Poster aux Journées ouvertes pour la Biologie, l’informatique et les Mathematiques JOBIM 2015. Clermont-Ferrant (France). 2 pages.
\item [C21]	Wollbrett J., Larmande P. and Ruiz M. Intégration automatique d’une ontologie de domaine dans un annuaire Biomoby. Presentation orale aux Journées ouvertes pour la Biologie, l’informatique et les Mathematiques JOBIM2009, Nantes (France). 8 pages.
\item [C22]	Larmande P., Tranchant C., Libourel T., Mougenot I. Intégration de données en génomique végétale. Journées Ouvertes à la Biologie, l’Informatique et les Mathématiques, Satellite Workshop Ontologie, Grille et Intégration Sémantique pour la Biologie à la conference Biologie, l’informatique et les Mathematiques JOBIM 2007. Clermont-Ferrant (France)JOBIM 2005, Lyon. 8 pages.
\end{itemize} 




